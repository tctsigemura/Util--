%
% Util-- の解説書
%
% 2016.10.09         : バージョン番号をUtil--全体で共通化
% 2016.10.09         : size-- が .exe を扱えるようになる
% 2016.03.12         : プログラミング言語C-- から分離して新規作成
%
% 
% $Id$
%
\documentclass[11pt,a4j,twoside,dvipdfmx]{jbook}
%----------------------------------------------------------------------
\usepackage{graphicx}
\usepackage{amsmath}
\usepackage{amssymb}
\usepackage{bm}
\usepackage{fancybox}
\usepackage{fancyhdr}
\usepackage{lastpage}
\usepackage[usenames]{color}
\usepackage{multicol}
\usepackage{bigfoot}  % \footnote{} 中で \verb/../ が利用可能になる。
%\usepackage{listings,jlisting}
\usepackage{url}

%----------------------------------------------------------------------
% ハイパーリンク設定(印刷用では色使いに注意)
\usepackage[colorlinks,linkcolor=blue,urlcolor=blue]{hyperref}  % PDF用
\usepackage{hyperref}                                           % 印刷用
\usepackage{pxjahyper}

%----------------------------------------------------------------------
\newcommand{\myfigure}[5]{
\begin{figure}[#1]
%\begin{figure}[tbp]
\begin{center}
\includegraphics[width=#2]{#3}
\vspace{-.3cm}
\caption{#4}
\label{fig:#5}
\end{center}
\end{figure}
}

%----------------------------------------------------------------------
\newcommand{\myfigureA}[5]{
\begin{figure*}[#1]
%\begin{figure*}[tbp]
\begin{center}
\includegraphics[width=#2]{#3}
\vspace{-.3cm}
\caption{#4}
\label{fig:#5}
\end{center}
\end{figure*}
}

%----------------------------------------------------------------------
\newcommand{\figref}[1]{図~\ref{fig:#1}}
\newcommand{\tabref}[1]{表~\ref{tab:#1}}
\newcommand{\ver}{Ver. 3.0.0}
\newcommand{\cmm}{{\tt C--}}
\newcommand{\cmmc}{{\tt c--}}
\newcommand{\cmml}{\cmm 言語}
\newcommand{\cl}{{\tt C}言語}
\newcommand{\javal}{{\tt Java}言語}
\newcommand{\tac}{TaC}
\newcommand{\tacos}{TacOS}
\newcommand{\util}{{\tt Util--}}
\newcommand{\as}{{\tt as--}}
\newcommand{\ld}{{\tt ld--}}
\newcommand{\objbin}{{\tt objbin--}}
\newcommand{\objexe}{{\tt objexe--}}
\newcommand{\size}{{\tt size--}}
\newcommand{\n}{{\tt $\backslash$n}}
\newcommand{\ul}{\_}
\newcommand{\lw}[1]{\smash{\lower2.0ex\hbox{#1}}}

%----------------------------------------------------------------------
\newenvironment{mylist}
{\begin{center}\begin{tabular}{|c|}\hline\\\begin{minipage}{0.9\textwidth}}
{\end{minipage}\\\\\hline\end{tabular}\end{center}}

%----------------------------------------------------------------------
\newenvironment{myminipage}
{\begin{minipage}{0.9\textwidth}\begin{center}\vspace{0.2cm}}
{\end{center}\vspace{0.2cm}\end{minipage}}

%----------------------------------------------------------------------
\newenvironment{mytable}[3]
{\small\begin{table}[#1]\begin{center}\caption{#2}\vspace{0.2cm}\label{tab:#3}}
{\end{center}\end{table}}

%----------------------------------------------------------------------
\begin{document}
%\setlength{\textwidth}{14cm}
%\setlength{\textheight}{22cm}
\setlength{\oddsidemargin}{10pt}
\setlength{\evensidemargin}{-10pt}
%\setlength{\topmargin}{10pt}
%\setlength{\marginparsep}{15pt}
%\setlength{\parskip}{0.3cm}
\setlength{\headsep}{1cm}
\frontmatter

%表紙
\title{\util 解説書 \\\ver}
\author{徳山工業高等専門学校\\情報電子工学科}
\date{}

\maketitle

% 著作権表示
\thispagestyle{empty}
~
\vfill
\begin{flushleft}
Copyright \copyright ~~ 2016 by \\
Dept. of Computer Science and Electronic Engineering, \\
Tokuyama College of Technology, JAPAN
\end{flushleft}

\vspace{0.8cm}

本ドキュメントは*全くの無保証*で提供されるものである.上記著作権者および
関連機関・個人は本ドキュメントに関して,その適用可能性も含めて,いかなる保証
も行わない.また,本ドキュメントの利用により直接的または間接的に生じたいかな
る損害に関しても,その責任を負わない.
\setcounter{page}{0}

% 目次
\tableofcontents

% 本文
\mainmatter
% 
% 1章 はじめに
%
\chapter{はじめに}

\util は\cmml を\tac で実行するために必要な5つのツールから構成されます。


\begin{itemize}
\item {\as}は\tac 用のアセンブラです。
アセンブラは{\cmm}コンパイラが出力したアセンブリ言語プログラムを、
リロケータブルオブオブジェクトに変換します。

アセンブリ言語の文法は、
「付録\ref{app:as} {\as}文法のまとめ」に簡単にまとめてあります。
現在のところ詳しいドキュメントがありません。
「文法のまとめ」の他には、{\as}の動作テストに使用される
\verb;Util--/AS--/test.s;ファイルが参考になります。

リロケータブルオブジェクトは、
機械語プログラム、名前表、再配置情報表からなるファイルです。
詳しくは、「付録\ref{app:oformat} {\tt .o} 形式ファイル」で説明します。

\item {\ld}は\tac 用のリンカです。
リンカは複数のリロケータブルオブジェクトを入力して、
一つのリロケータブルオブジェクトに結合します。

\item {\objexe}は、
リロケータブルオブジェクトを入力し、
{\tacos}のアプリケーションプログラムの実行形式ファイルを出力します。
実行形式ファイルは、リロケータブルオブジェクトから、
シンボルテーブルなどリンカしか使用しない情報を取り除いたファイルです。
実行形式ファイルについては「付録\ref{app:eformat}
{\tt .exe} 形式ファイル」で説明します。

\item {\objbin}は\tac 用のローダです。
ローダはリロケータブルオブジェクトを入力して、
ロードアドレスが決定された機械語を出力します。
出力ファイルの形式については「付録\ref{app:bformat}
{\tt .bin} 形式ファイル」で説明します。
ローダは、{\tacos}のカーネルを作成する時に使用されます。

\item {\size}は、
リロケータブルオブジェクトのテキストセグメント、
初期化データセグメント、
非初期化データセグメントの大きさを表示します。
完成したプログラムのメモリ使用量を見積もるために使用します。
\end{itemize}
 % はじめに
% 
% 2章 Util--のインストール
%
\chapter{\util のインストール}

\util は\url{https://github.com/tctsigemura/Util--/}から入手します。
インストールの準備ができたらダウンロードした配布物を解凍し
\verb/Util--/ディレクトリで以下のように操作します。

\begin{mylist}
\begin{verbatim}
$ make
...
cc -std=c99  -DDATE="\"`date`\"" -DVER=\"2.1.1\" \
	-o as-- syntax.c lexical.c util.c
cc -std=c99 -DDATE="\"`date`\"" -DVER=\"2.0.0\" -DARC=\"TaC\"
  -o ld-- ld.c
cc -std=c99 -o objbin-- objbin.c
cc -std=c99 -o size-- size.c
cc -std=c99 -o objexe-- objexe.c
$ sudo make install
...
install -d -m 755 /usr/local/bin
install -m 755 as-- /usr/local/bin
...
$
\end{verbatim}
\end{mylist}

以上で、\as 、\ld 、\objbin 、\objexe 、\size の五つのプログラムが
コンパイルされ、\verb;/usr/local/bin;にインストールされました。
 % インストール
% 
% $Id$
%
\chapter{コマンドリファレンス}

\util に含まれる5つのツールの使用方法を解説します。

\section{{\as}コマンド}

{\as}コマンドは{\tac}用のアセンブラです。
{\tac}のアセンブリ言語で記述されたプログラムを
リロケータブルオブジェクトに変換します。
リロケータブルオブジェクトファイルの形式は、
「付録\ref{app:oformat} \verb/.o/形式ファイル」で解説してあります。

{\as}には手書きのプログラムを入力することもできますが、主に、
{\cmm}コンパイラが出力したアセンブリ言語プログラムを
入力することを想定しています。
そのため、オペランドにアドレス計算式が書けない等、
{\cmm}コンパイラが使用しない機能は省略されています。
{\tac}アセンブリ言語の文法は「付録\ref{app:as} \as 文法まとめ」の通りです。

{\as}コマンドの書式は次の通りです。

\begin{flushleft}
{\bf 形式 : }~~~\verb/as-- [-h] [-v] [<ソースプログラムファイル>]/
\end{flushleft}

\verb/-h/、\verb/-v/オプションは使用法メッセージを表示します。
ソースプログラムファイルの拡張子は「\verb/.s/」でなければなりません。
{\as}コマンドは拡張子が「\verb/.o/」のファイルを作成し、
再配置可能な機械語を出力します。
次のように実行すると、``\verb/hello.o/''ファイルが作成されます。

\begin{mylist}
\begin{verbatim}
$ as-- hello.s
\end{verbatim}
\end{mylist}

\section{{\ld}コマンド}

{\ld}コマンドは{\tac}用のリンカープログラムです。
「\verb/.o/形式ファイル」(\pageref{app:oformat}ページ参照)を複数入力し、
同じ形式の一つのファイルに結合します。
{\ld}コマンドの書式は次の通りです。
出力ファイルは一つだけ、
入力ファイルはいくつでも指定できます。

\begin{flushleft}
{\bf 形式 : }~~~\verb/ld-- [-h] [-v]  <出力ファイル> <入力ファイル> .../
\end{flushleft}

次のように実行すると``\verb/libtac.o/''、``\verb/hello.o/''の
二つのファイルを結合して、``\verb/mod.o/''ファイルを作成します。
``\verb/hello.sym/''ファイルは出力されたリロケータブルオブジェクトの
再配置表と名前表をダンプしたファイルです。

``\verb/mod.o/''ファイルもリロケータブルオブジェクトなので、
未定義シンボルを含んでいる可能性があります。

\begin{mylist}
\begin{verbatim}
$ ld-- mod.o /usr/local/cmmLib/libtac.o hello.o > hello.sym
\end{verbatim}
\end{mylist}

\section{{\objexe}コマンド}

{\objexe}コマンドは{\tacos}用の実行形式ファイル作成プログラムです。
「\verb/.o/形式ファイル」(\pageref{app:oformat}ページ参照)を入力し、
「\verb/.exe/形式ファイル」(\pageref{app:eformat}ページ参照)へ変換します。
{\objexe}コマンドの書式は次の通りです。
入力ファイルも出力ファイルも一つだけ指定できます。

\begin{flushleft}
{\bf 形式 : }~~~\verb/objexe-- <exefile> <objfile> <stkSiz>/
\end{flushleft}

``\verb/<exefile>/''は出力のファイル名です。
``\verb/<objfile>/''は入力のファイル名です。
``\verb/<stkSiz>/''には、
アプリケーションプログラムに割り付ける
「スタック領域サイズ+ヒープ領域サイズ」をバイト単位の10進数で指定します。

下に実行例を示します。
\verb/mod.o/は、\ld が出力したリロケータブルオブジェクトです。
\verb/hello.map/ファイルには、
\verb/hello.exe/中のシンボルとアドレスの対応表がアドレス順に格納されます。
シンボルのアドレスは、アプリケーションプログラムの先頭からの相対アドレスです。

\begin{mylist}
\begin{verbatim}
$ objexe-- hello.exe mod.o 600 | sort --key=1 > hello.map
\end{verbatim}
\end{mylist}

\verb/mod.o/に未解決シンボルが含まれているとエラーになります。
次の実行例は\verb/hello.s/ファイル
(\verb/hello.cmm/から\cmmc が変換したアセンブリ言語ソースプログラム)中に、
未定義シンボル\verb/_printff/(\cmm ソース中では\verb/printff/)が
含まれていてエラーになった例です。

\begin{mylist}
\begin{verbatim}
$ objexe-- hello.exe mod.o 600 | sort --key=1 > hello.map
_printff[hello.s]:undefined symbol
\end{verbatim}
\end{mylist}

\section{{\objbin}コマンド}

{\objbin}コマンドは{\tac}用のローダプログラムです。
「\verb/.o/形式ファイル」(\pageref{app:oformat}ページ参照)を入力し、
「\verb/.bin/形式ファイル」(\pageref{app:bformat}ページ参照)へ変換します。
{\objbin}コマンドの書式は次の通りです。
入力ファイルも出力ファイルも一つだけ指定できます。

\begin{flushleft}
{\bf 形式 : }~~~\verb/objbin--  0xTTTT <出力ファイル> <入力ファイル>  [0xBBBB]/
\end{flushleft}

\verb/0xTTTT/は16進数でテキストセグメントの先頭アドレスを指定します。
指定した番地からテキストセグメント、データセグメントの順に配置されます。
\verb/0xBBBB/は16進数でBSSセグメントの先頭アドレスを指定します。
\verb/0xBBBB/は省略可能です。
省略した場合、BSSセグメントはデータセグメントの直後に配置されます。

下に実行例を示します。
実行例の1行目は``\verb/kernel.o/''ファイルを入力し\verb/0000H/番地に配置した
状態の機械語を作成し``\verb/kernel.bin/''ファイルに出力します。
実行例の2行目は``\verb/ipl.o/''ファイルを入力し\verb/F000H/番地に
テキストセグメントとデータセグメントを連続して配置し、
\verb/DA00H/番地に BSSセグメントを配置します。
そして、その状態の機械語を``\verb/ipl.bin/''ファイルに出力します。
なお、これらの例は実物のカーネルとIPLを配置する手順です。
\verb/F000H/番地からの始まる ROM 領域に IPL プログラム、
RAM 領域の\verb/DA00H/番地にワークエリアを配置します。

\begin{mylist}
\begin{verbatim}
$ objbin-- 0x0000 kernel.bin kernel.o | sort --key=1 > kernel.map
$ objbin-- 0xf000 ipl.bin ipl.o 0xda00 | sort --key=1 > ipl.map
\end{verbatim}
\end{mylist}

{\objbin}コマンドは、固定番地に配置された完全な機械語を作成します。
``\verb/kernel.o/''ファイルや``\verb/ipl.o/''ファイルに未定義シンボルが
含まれていると完全な機械語に変換できないのでエラーが発生します。

%次の実行例はエラーが発生する例です。
%実行例のエラーメッセージは、
%``\verb/kernel.s/''ファイル中で参照された \verb/_printf/
%と言うシンボルが未定義であることを表しています。
%
%\begin{mylist}
%\begin{verbatim}
%$ objbin-- 0x0000 kernel kernel.o
%_printf[kernel.s]:undefined symbol
%\end{verbatim}
%\end{mylist}

\section{{\size}コマンド}

{\size}コマンドは
「\verb/.o/形式ファイル」(\pageref{app:oformat}ページ参照)を入力し、
テキスト、データ、BSSセグメントのサイズを表示します。

\begin{flushleft}
{\bf 形式 : }~~~\verb/size--  <入力ファイル>/
\end{flushleft}

次のように実行します。
``\verb/kernel.o/''ファイルの各セグメントと全体のサイズが
10進数と16進数で表示されています。

\begin{mylist}
\begin{verbatim}
$ size-- kernel.o
text    data    bss     dec     filename
17436     516    4386   22338   kernel.o
(441c)  (0204)  (1122)  (5742)  (hex)
\end{verbatim}
\end{mylist}
 % コマンドリファレンス

\appendix
\include{syntax}  % 付録 アセンブラ文法まとめ
\include{binfile} % 付録 ファイルフォーマット

% 発行元など
\backmatter
\pagestyle{empty}
\onecolumn
~
\subsubsection{変更履歴}
\begin{flushleft}
2016年10月09日 {\size}の仕様変更、{\util}全体でバージョン番号統一  \\
2016年03月13日 初期バージョン  \\
\end{flushleft}

\vfil

\subsubsection{対応ソフトウェアのバージョン}
\begin{tabular}{|l|l|}
\hline
{\util}   & {\ver} \\
\hline
\end{tabular}

\vfill\vfill
\begin{center}
\fbox{\parbox{10cm}{ \vspace{0.3cm}
\large{\bf \util 解説書} \\
\\
 発行年月 2016年10月 \ver \\
 発  行 独立行政法人国立高等専門学校機構 \\
      徳山工業高等専門学校 \\
      情報電子工学科 重村哲至 \\
      〒745-8585 山口県周南市学園台 \\
      sigemura@tokuyama.ac.jp \\
}}
\end{center}

\begin{center}
\end{center}
\vfill
\end{document}
